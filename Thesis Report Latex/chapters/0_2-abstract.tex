\chapter*{Abstract}
The rate of publication of scientific papers has grown exponentially in recent years. As a result, it has become increasingly difficult for a researcher to find papers related to their research topic that are 'cite-worthy'. As in search engines, e-commerce and other fields which have had to deal with the information explosion problem, recommender systems have come to the rescue. Citation recommendation systems, which have been actively researched in the last few years, recommend citations for either a complete paper or a small portion of text called a \textit{citation context}. The process of recommending citations for citation contexts is called local citation recommendation, and is the focus of this thesis.
The Microsoft Academic Graph (MAG) data set, which has not been used in the local citation recommendation setting so far, is used in this thesis to train models using several algorithms based on embeddings, topic modelling and information retrieval techniques. The MAG data set is a rich data set containing both citation data and paper metadata, making it ideal for this task. 
The models created using the aforementioned algorithms are evaluated extensively offline using the MAG and other data sets created expressly for that purpose. Another data set created from the MAG helps to prove that citation contexts describe the cited paper more than the citing paper.
The best-performing algorithms are combined into a hybrid algorithm, which is evaluated offline as well as online by conducting a user study. The results show that a hybrid model containing embedding and information retrieval-based components outperforms its individual components by a large margin, and outperforms the other algorithms used in the thesis as well. This hybrid model is used to create a running recommender system, which is made available to the public.

\chapter*{Zusammenfassung}
Die Veröffentlichungsrate von wissenschaftlichen Arbeiten ist in den letzten Jahren exponentiell gestiegen. Infolgedessen ist es für einen Forscher immer schwieriger geworden, zu seinem Forschungsthema passende Arbeiten zu finden, die „zitierwürdig“ sind. Wie in Suchmaschinen, E-Commerce und anderen Bereichen, die sich mit dem Problem der Informationsexplosion befassen mussten, sind Empfehlungssysteme zur Rettung gekommen. Zitierempfehlungssysteme, die in den letzten Jahren aktiv erforscht wurden, empfehlen Zitate für eine vollständige Arbeit oder einen kleinen Teil des Textes, der als \textit{Zitierkontext} bezeichnet wird. Der Prozess der Empfehlung von Zitaten für Zitierkontexte wird als lokale Zitierempfehlung bezeichnet und steht im Mittelpunkt dieser Arbeit.
Der Microsoft Academic Graph (MAG)-Datensatz, der bisher nicht für lokale Zitierempfehlung verwendet wurde, wird in der Arbeit zum Trainieren von Modellen mit verschiedenen Algorithmen verwendet, die auf Einbettungen, Themenmodellierung und Information Retrieval-Techniken basieren. Der MAG-Datensatz ist ein umfangreicher Datensatz, der sowohl Zitierdaten als auch Papiermetadaten enthält und sich ideal für diese Aufgabe eignet.
Die mit den genannten Algorithmen erstellten Modelle werden weitgehend offline mit dem MAG und anderen eigens dafür erstellten Datensätzen ausgewertet. Ein anderer Datensatz, der aus dem MAG erstellt wurde, hilft zu beweisen, dass Zitierkontexte das zitierte Papier mehr beschreiben als das zitierte Papier.
Die leistungsstärksten Algorithmen werden zu einem Hybridalgorithmus kombiniert, der durch eine Anwenderstudie sowohl offline als auch online ausgewertet wird. Die Ergebnisse zeigen, dass ein Hybridmodell, das eingebettete und auf dem Abruf von Informationen basierende Komponenten enthält, seine einzelnen Komponenten um ein Vielfaches übertrifft und auch die anderen in der Arbeit verwendeten Algorithmen übertrifft. Mit diesem Hybridmodell wird ein laufendes Empfehlungssystem erstellt, das der Öffentlichkeit zugänglich gemacht wird.