\chapter{Conclusion}\label{chap:conclusion}

In this thesis, the field of local citation recommendation was explored using the Microsoft Academic Graph data set, and a hybrid recommender system was created. Five major algorithms were investigated: 4 based on embeddings (Doc2Vec, Paper2Vec and 2 HyperDoc2Vec algorithms), 1 based on topic modelling (LDA) and 1 based on infromation retrieval (BM25). 
To this end, six data sets were created and made public. The main data source, MAG, does not provide full text. So, a data set was created wherein the title, abstract and citation contexts were combined to form what is referred to in this thesis as 'pseudo full-text'. A restricted version of the MAG data set containing only papers with 50 or more citations (MAG50) was also created. In addition to this, full text from 3 other data sources was combined with pseudo full-text from MAG, and used for evaluation.

By conducting an offline evaluation on large test sets, it was found that 2 of the embedding algorithms, Doc2Vec and Paper2Vec, do a pretty poor job at citation recommendation. LDA, while it performs marginally better, still does a poor job on all the metrics. The Hyperdoc2vec model produces two different embeddings for each paper -- IN embeddings model the paper's role as a citing paper, and OUT embeddings model the paper's role as a cited paper. Through our offline and online evaluation, we found that using only the OUT embeddings (called hd2vOUT in the thesis) for prediction produced the best scores across metrics. Using both vectors resulted in worse evaluation scores. The BM25 algorithm, which is based on text matching, compared well with most of the other algorithms.

While hd2vOUT outperformed BM25 on the evaluations using the full text-based data sets, the two algorithms were very close while evaluating on MAG, especially on the Recall metric.

By doing a user study, it was found that hd2vOUT often produces results that are more general, while BM25 produces more specific results. Thus, they complemented each other well. 

Recommendations from both algorithms were combined stochastically using a semi-genetic hybrid recommendation algorithm. This hybrid algorithm/model drastically outperformed its components on all metrics and data sets in the offline evaluation results. The user study compared the hybrid model's recommendations with those of the individual BM25 and hd2vOUT models. It was found that while there were cases where the most relevant results from the individual algorithms were missing from the hybrid model's recommendations, this did not happen often. The hybrid model's recommendations generally tended to produce more relevant results than either of the individual models.

In addition, it was found that the hybrid model satisfied the serendipity requirement of recommender systems, where papers were recmmended which might be interesting to the user, even if they are not the papers they would want to recommend.

By performing a case study, it was demonstrated that using citing paper's citation contexts as part of a cited paper's content improved the performance of BM25 many times over. A separate data set called MAG-Cited was created for this purpose. Each paper's content in this data set was its title, abstract and citation contexts from papers which cite it. The excellent performance of BM25 on MAG-Cited suggested the possibility of creating a second hybrid model, based on two data sets and three components. This hybrid model, called Hybrid23, performed much better than the original hybrid model (whose two components are the first two components of Hybrid23) and marginally better than the its third component, BM25 based on MAG-Cited.
The Hybrid23 model was finally plugged into a running recommender system, where the user can enter his/her citation context and get recommendations. A second hybrid model based on the MAG50 data set was also used under the hood of the recommender system to give the user another option while getting recommendations.

Overall, it was clear that citation recommendation is a far from simple task, as borne out in both the offline and online evaluation results. The hybrid algorithms performed much better than their individual components, but they can be improved by improving their components. Using a hybrid algorithm based on two distinct MAG data sets created a much better citation recommendation system as the data was more representative.